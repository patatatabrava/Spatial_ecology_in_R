\documentclass{beamer}
\usepackage{graphicx} % Required for inserting images
\usepackage{hyperref}
\usepackage{animate}

\usetheme{Frankfurt}
\usecolortheme{seahorse}

\title{Spotting the spotted owl}
\author{Cloé Rouch}
\date{\today}

\begin{document}

\maketitle

\section{Introduction}

\begin{frame}{The spotted owl}
    \begin{columns}
        \column{.3\textwidth}
            \includegraphics[width =\textwidth]{Northern_Spotted_Owl.USFWS.jpg} 
        \column{.5\textwidth}
            \begin{itemize}
                \item Habitat: old-growth conifer forest
                \item What happened to its habitat?
                    \begin{itemize}
                        \item until the '90s, the old forest was cut by logging companies
                    \end{itemize}
                $\implies$ Threatened status 
            \end{itemize}
            \begin{itemize}
                \item Biggest risk now: competition with barred owls
                \item How to monitor its population?
                    \begin{itemize}
                        \item direct observation
                        \item mapping out potential habitat
                    \end{itemize}
            \end{itemize}
        \column{.3\textwidth}
            \includegraphics[width =\textwidth]{Strix-varia-005.jpg} 
    \end{columns}    
\end{frame}

\section{Visualizing direct observations}

\subsection{Data}

\begin{frame}{Downloading and loading the data}
    \begin{itemize}
        \item Source of the data: Global Biodiversity Information Facility (GBIF)
        \item Downloaded using the website's interface:
            \begin{itemize}
                \item searched for \textit{Strix occidentalis caurina}
                \item selected the iNaturalist research-grade observations (they have coordinates)
                \item downloaded the simple version of the data
            \end{itemize}
        \item DOI of the dataset: 10.15468/dl.rx8u8v
    \end{itemize}
    \includegraphics[width = \textwidth]{Code/Loading the data.png}
\end{frame}

\begin{frame}{Cleaning up the data}
    \includegraphics[width = \textwidth]{Code/Cleaning up the data.png}
\end{frame}

\subsection{Base map}

\begin{frame}{Getting and treating the data on the states to map out}
    \includegraphics[width = \textwidth]{Code/States data.png}
\end{frame}

\begin{frame}{Generating the base map}
    \begin{columns}
        \column{.8\textwidth}
            \includegraphics[width = \textwidth]{Code/Base map.png}
        \column{.3\textwidth}
            \includegraphics[width = \textwidth]{The map itself.png}
    \end{columns}    
\end{frame}

\begin{frame}{Adding the data to the base map}
    \includegraphics[width = \textwidth]{Code/Map and data.png}
    \begin{columns}
        \column{.2\textwidth}
        \column{.5\textwidth}
            This map shows observations from all years of the dataset:
        \column{.5\textwidth}
            \includegraphics[width = .5\textwidth]{Map with all years.png}
    \end{columns}
\end{frame}

\begin{frame}{Animating the map}
    \begin{columns}
        \column{.6\textwidth}
            \includegraphics[width = \textwidth]{Code/Anim without shadow.png}
        \column{.5\textwidth}
           \animategraphics[loop, autoplay, width = \textwidth]{2}{without_shadow/without_shadow-}{0}{30}
    \end{columns}
\end{frame}

\begin{frame}{Keeping the shadow of previous points}
    \begin{columns}
        \column{.6\textwidth}
            \includegraphics[width = \textwidth]{Code/Anim with shadow.png}
        \column{.5\textwidth}
            \animategraphics[loop, autoplay, width = \textwidth]{2}{with_shadow/with_shadow-}{0}{30}
    \end{columns}
\end{frame}

\section{Monitoring habitat}

\begin{frame}{Method and source of the data}
    \begin{itemize}
        \item The DVI can be used to determine the type of trees in a forest: low $\implies$ conifers, high $\implies$ hardwood trees
        \item Source of the data: Earth Explorer website
        \item Downloaded using the website's interface (see resource on Virtuale)
        \item Data: 
        \begin{itemize}
            \item bands 4 (red) and 5 (NIR) of the Landsat 8 satellite
            \item January of 4 years (2018, 2019, 2021, 2022)
            \item patch of forest at the border between Oregon and California
        \end{itemize}
    \end{itemize}
\end{frame}

\begin{frame}{Calculating the NDVI for each year}
    \includegraphics[width = \textwidth]{Code/Calculating the NDVI.png}
\end{frame}

\begin{frame}{Plotting the results}
    \includegraphics[width = \textwidth]{Code/Plotting the results.png}
    \centering
    \includegraphics[width = .8\textwidth]{Multiframe NDVI.png}
\end{frame}

\section{References}

\begin{frame}{}

    \centering {\Large Thank you for your attention!}

    \bigskip
    
    \raggedright References:
    \begin{itemize}
    \item \href{https://earthobservatory.nasa.gov/features/SpottedOwls/spotted_owls.php}{https://earthobservatory.nasa.gov/features/SpottedOwls/ spotted\_owls.php}
    \item \href{https://earthobservatory.nasa.gov/images/145979/spotting-the-spotted-owl-30-years-of-habitat-change}{https://earthobservatory.nasa.gov/images/145979/spotting-the-spotted-owl-30-years-of-habitat-change}
    \item \href{https://earthobservatory.nasa.gov/images/146038/spotting-the-spotted-owl-30-years-of-forest-disturbance}{https://earthobservatory.nasa.gov/images/146038/spotting-the-spotted-owl-30-years-of-forest-disturbance}
    \item \href{https://conservancy.umn.edu/bitstream/handle/11299/220339/time-maps-tutorial-v2.html?sequence=3&isAllowed=y}{https://conservancy.umn.edu/bitstream/handle/11299/ 220339/time-maps-tutorial-v2.html?sequence=3\&isAllowed=y}
\end{itemize}
\end{frame}

\end{document}
